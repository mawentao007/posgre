

\documentclass{beamer}
\usepackage[utf8]{inputenc}
\usepackage{CJK}
\usepackage{verbatim}
\usepackage{graphicx}




\begin{document}
\begin{CJK*}{UTF8}{gkai}
\title{redis数据类型} 
\author{马文韬} 
\date{\today} 

\frame{\titlepage} 

%\frame{\frametitle{Table of contents}\tableofcontents} 


\section{SDS} 
\begin{frame}[fragile]
\frametitle{SDS}
SDS数据类型表示
\begin{verbatim}
	typedef char *sds;
	struct sdshdr{
		int len(buf总长度);
		int free(未使用空间大小);
		char buf[];
	}
\end{verbatim}
\end{frame}


\begin{frame}
\frametitle{SDS}
SDS类型的作用: 实现字符串对象(StringObject)
\end{frame}

\begin{frame}
\frametitle{SDS}
SDS的扩容
\begin{itemize}
\item  如果所需空间小于1M,则分配所需空间的2倍大小。所需空间就是将字符串放入buf之后的大小。
\item  如果所需空间大于1M,额外分配1M空间大小。
\end{itemize}
\end{frame}

\begin{frame}
\frametitle{SDS}
SDS的优点
\begin{itemize}
\item 增加追加操作的速度
\item 计算字符串长度时间缩短
\end{itemize}

\end{frame}


\begin{frame}[fragile]
\frametitle{DICT}
字典数据类型
\begin{verbatim}
typedef struct dictEntry {
    void *key;
    union {
        void *val;
        uint64_t u64;
        int64_t s64;
    } v;
    struct dictEntry *next;
} dictEntry;
\end{verbatim}

\end{frame}

\begin{frame}[fragile]
\frametitle{DICT}
\begin{verbatim}
typedef struct dictht {
    dictEntry **table;
    unsigned long size;
    unsigned long sizemask;  //长度掩码,用于计算索引
    unsigned long used;
} dictht;
\end{verbatim}
\end{frame}

\begin{frame}[fragile]
\frametitle{DICT}
\begin{verbatim}
typedef struct dict {
    dictType *type;
    void *privdata;
    dictht ht[2];
    int rehashidx; /*not  rehashing  rehashidx == -1 */
    int iterators; /* number of iterators  */
} dict;
\end{verbatim}
\end{frame}

\begin{frame}[fragile]
\frametitle{DICT}
\begin{verbatim}
typedef struct dictType {
    unsigned int (*hashFunction)(const void *key);
    void *(*keyDup)(void *privdata, const void *key);
    void *(*valDup)(void *privdata, const void *obj);
    int (*keyCompare)(void *privdata, const void *key1, const void *key2);
    void (*keyDestructor)(void *privdata, void *key);
    void (*valDestructor)(void *privdata, void *obj);
} dictType;
\end{verbatim}
\end{frame}

\begin{frame}
\frametitle{DICT}
\begin{figure}[ht!]
\centering
\includegraphics[width=100mm]{dict.png}
\caption{Dict 结构图}
\label{overflow}
\end{figure}

\end{frame}


\begin{frame}
\begin{center}
 END
\end{center}
\end{frame}

% \begin{frame}
% \end{frame}
% 
% \begin{frame}
% \end{frame}
% 
% \begin{frame}
% \end{frame}
% 
% \section{DICT} 
% 
% \begin{frame}[fragile]
% \begin{itemize}
% \item Introduction to  \LaTeX  
% \item Course 2 
% \item Termpapers and presentations with \LaTeX 
% \item Beamer class
% \end{itemize} 
% \end{frame}
% 
% \frame{\frametitle{lists with pause}
% \begin{itemize}
% \item Introduction to  \LaTeX \pause 
% \item Course 2 \pause 
% \item Termpapers and presentations with \LaTeX \pause 
% \item Beamer class
% \end{itemize} 
% }
% 
% \subsection{Lists II}
% \frame{\frametitle{numbered lists}
% \begin{enumerate}
% \item Introduction to  \LaTeX  
% \item Course 2 
% \item Termpapers and presentations with \LaTeX 
% \item Beamer class
% \end{enumerate}
% }
% \frame{\frametitle{numbered lists with pause}
% \begin{enumerate}
% \item Introduction to  \LaTeX \pause 
% \item Course 2 \pause 
% \item Termpapers and presentations with \LaTeX \pause 
% \item Beamer class
% \end{enumerate}
% }
% 
% \section{Section no.3} 
% \subsection{Tables}
% \frame{\frametitle{Tables}
% \begin{tabular}{|c|c|c|}
% \hline
% \textbf{Date} & \textbf{Instructor} & \textbf{Title} \\
% \hline
% WS 04/05 & Sascha Frank & First steps with  \LaTeX  \\
% \hline
% SS 05 & Sascha Frank & \LaTeX \ Course serial \\
% \hline
% \end{tabular}}
% 
% 
% \frame{\frametitle{Tables with pause}
% \begin{tabular}{c c c}
% A & B & C \\ 
% \pause 
% 1 & 2 & 3 \\  
% \pause 
% A & B & C \\ 
% \end{tabular} }
% 
% 
% \section{Section no. 4}
% \subsection{blocs}
% \frame{\frametitle{blocs}
% 
% \begin{block}{title of the bloc}
% bloc text
% \end{block}
% 
% \begin{exampleblock}{title of the bloc}
% bloc text
% \end{exampleblock}
% 
% 
% \begin{alertblock}{title of the bloc}
% bloc text
% \end{alertblock}
% }
\end{CJK*}
\end{document}

