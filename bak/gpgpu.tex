本文的主要目标是展示一种可以在通用图像处理单元(gpgpu)上使用的调度策略,这个策略可以协调线程调度和预取指令,更好的处理内存延迟的状况。

传统的warp调度策略没法有效的协同预取数据的原因:
	主要原因是调度策略连续调度warp,warps一般都会访问附近的cache blocks。这样会造成两个后果:
		预取指令由当前warp发出,另一个warp就请求相应的地址;
		要设计非常复杂的预取策略才能预测“far-ahead”的warp所需要地址。

我们的调度策略的核心思想:
	在时间上分开调度连续的warp,这样的话他们就不会一个挨着一个的执行.

	这个调度策略可以和简单的预取策略协同工作,容忍内存延迟;
	即使没有预取策略,也可以促进存储器组并行化。


通常,GPGPUs通过同时执行很多线程来避免内存延迟。固定数量的线程被分为一组,称为warp,同一个warp中的线程共享相同的指令流,执行相同的指令,这就是single instruction multiple threads,SIMT。
高水平的warp调度实现快速的上下文切换,当一个warp进行长时间访存操作,另一个warp可以执行,这样来隐藏延迟。
warp调度策略决定了warp的执行顺序和时间,在访存延迟的隐藏和内存带宽的利用上都有重要作用。

	
